\section{Related Work}
Bürkle et. al proposed the idea of a swarm of drones fulfilling a task rather than a single drone in their article \textit{Towards Autonomous Micro UAV Swarms}\cite{burkleetal}. They suggested a system for inter-drone cooperation to seamlessly orchestrate several drones with different types of sensors in an environment. They also utilized simulation to test their system. This system is far larger than the produced system and encompasses several other aspects that we do not consider. It provides a basis for our simulation-based development efforts.

Sima Mitra presented an Autonomous Quadcopter Docking System\cite{simamitra} that allowed drones to return to a charging station using computer vision for locating the docking station. In it, she showed that autonomously returning to a charging station is fully possible without a locationing system. Although our framework expects some kind of locationing system for the airborne drones, this is useful as it demonstrates the automated landing on a specified target as battery depletes. The inherent limits of the reliance on computer vision are, however, not desirable.

Both Altug et al.\cite{altugetal} and Ducard et. al.\cite{ducardetal} demonstrated the use of visual feedback for positioning and autonomous control of a drone. This is useful for the practical part of this project in which a kinect will be used to track drones in order to autonomously perform tasks.

Also relevant is the work of The Intelligent Mechatronics Lab at Boston University, in which a conductive surface is used to charge a drone without human interaction\cite{bostonuni}. This is of paramount importance in terms of realizing this solution, as full automation will not be possible unless the drone can charge itself without human interference.



\todo{Something about what we're doing different.}