\section{Introduction}
Aerial drones have a lot of potential use, but the use of commercially available drones are limited by the short amount of time they can stay airborne before requiring a recharge. Many applications could be considered where a drone would need to stay airborne for long periods of time to do various tasks. For instance, monitoring areas using various sensors or spreading pesticide on a field of crops.

Although battery life will increase as technology advances, there is a current need for alternatives. If drones could recharge themselves or somehow become sustainable this would increase the usage scenarios of drones significantly. Our idea is not to have drones stay airborne indefinitely, but to have the tasks that the drones are performing continue in spite of the need for a recharge. This could either be done by having other fully-charged drones take over the task or by allowing the drone to resume its task after it has been recharged. In this report, we outline a framework aimed at simplifying this functionality for drone-application developers.