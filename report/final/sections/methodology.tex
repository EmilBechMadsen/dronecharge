\section{Methodology}
The project was developed in two parts; a practical part and a theoretical part. The theoretical part consisted of a simulation of an \textit{ideal} drone, with which the framework was incrementally developed. The practical part consisted of a visual locationing system using Microsoft Kinect\footnote{http://www.microsoft.com/en-us/kinectforwindows/}\footnote{https://github.com/OpenKinect/libfreenect}\,\cite{benavidez2011mobile}, with which a quadcopter drone was controlled using the native SDK. Initially, a Crazyflie\footnote{http://www.bitcraze.se/crazyflie/} drone was used for the practical part, but after some difficulties with stability, it was decided utilize a Wizard-of-Oz\,\cite{kelley1984iterative} approach for drone movement. This allowed us to focus on the framework itself rather than implementation issues of a single type of drone. Once the framework was in place, we were offered the use of an AR Drone\,\footnote{http://ardrone2.parrot.com/}. This enabled us to evaluate our solution through the implementation of  a driver and associated tasks for a specific drone.

While others such as Mitra and Ducard et al. before us had attempted to do various parts of this project in isolation, DroneCharge brought together the components in order to create a practical solution on a more approachable level. The swarm concept presented by B\"urkle et al. also utilized many of the same concepts, but in a very large-scale and hardware-expensive fashion out of reach of the common developer. We relied on the various proof of concepts shown by our predecessors and envisioned a system in which the various components were useful without being out of reach for the average developer in terms of hardware and time.